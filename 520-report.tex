\documentclass{article}

\usepackage{graphicx}
\usepackage{url}
\usepackage{appendix}


\title{Improved Network Map Display}
\author{Joel Oughton}
\date{}

\begin{document}

\maketitle
%\tableofcontents

\begin{abstract}
  % Adjust my old abstract to reflect overall project state.

\end{abstract}

\section{Introduction}
\label{sec:introduction}
  % Motivation for my project
  % List the achievements (refer to sections)

\section{Related Work}
\label{sec:related-work}
  % ** General Related Work Topics **
  %   Web graphics sclability 
  %   Web based visualisation issues
  %   Computer network visualisation issues
  %   Published implementations
  %
  % Talk about these areas how there is no work that combines all this.
  %
  % 

  % Computer network visualisation (key issues)
  % 
  % Key Issues: 
  %   * Why use them
  %   * Core visualisation issues 
  %   * Why should it be interactive? 
  %   * How to use hierachy to our advantage
  %
  %  -> Set the scene in terms of - why visualise networks
  %  -> Core visual issues (Eick 1996)
  %  -> The importance and possibility of interactivity in visuals over the web (Alves, Becker)
  %  -> Hierachies (Eick) X
% why use them...

The related work to this project can be divided into three sections. Issues
relating to computer network visulisation, web based displays and actual network
map implementations. Computer network visualisation work includes the key issues
relating to computer network visualisation such as presentation ideas,
techniques to address network scalability problems and interactivity
considerations.

Eick describes aspects of network visualisation and identifies stengths and
weaknesses of graph based displays.\cite{Eick_1996} Network maps are most
commonly visualised using node and link graphs. It is noted that these maps are
particularly effective for small and sparse networks. Problems arrise with
larger networks such as display clutter, node positioning difficulties and
perceptual tension. Eick presents three strategies to address these problems.
Dynamic parameter focussing, node positioning and the use of a 3D layout. This
project considers the first two but only uses 2D layouts. While it is possible
that the visualisation techniques that are presented could be applied to a 3D
layout, it is considered outside the scope of this project and left to future
work.

% interactive Difficult to follow flows through network.  Many intersecting
% lines - difficult to interpret Hard to come up with good heuristic for
% visualisation parameters such as line width, node size etc
Becker, \emph{et al.} look in to the limitations of static network maps and
explain the benefits of adding dynamic features.\cite{Becker_1990} They recognise the difficulty
of comming up with good hueristics for determining visual parameters such as
line width and node sizing. Dynamic maps have the advantage of incoporating
sliders or other controls that allow parameters to be adjusted and viewed in
real time. It is often useful to follow paths through an network. This becomes
very hard to do for larger, more cluttered networks using a static map. The
design of the network map for this project considers these advantages as well as
the benefits from being able to make use latest interactivity technologys
available in the browser.


  % Web Visualisation
  %  
  % Why visualise on the web?  What to look out for?  Whats out there? (or is
  % this too similar to published implementations?)
  %
  % 
  % VMRL - similar ideas (detail added on zoom etc) Rohrer (1997)
  %
Rohrer and Swing saw the benefits of web based visualisations back in
1997.\cite{Rohrer_1997} They state that the loose coupling between data, users
and web applications is likely to provide a flexible medium for information
visualisation applications.  A side project mentioned in the paper is a web
based network visualisation tool that uses translucency for detail abstraction
and hyperlinks to explorer nodes in more detail. This tool was developed over 10
years ago and does not include the advanced HTML5 features available today.


  % \cite{Johnson_2008}
  %
  % HTML Native, Canvas, SVG.  Why do I need to consider this?  What do their
  % results show?  What does this mean for me?
  % 
It is important to pick a web graphics technology that scales well as the number
of nodes and edges increases in a network map. Johnson and Kelly performed a
scalability study on web-native information visualisation.\cite{Johnson_2008}
They test and analyse the scalability performance of the three most popular 2D
web graphics technologies which are SVG, Canvas and native HTML. The results
show that Canvas performs the best out of all three technologies. The authors
also note that neither SVG or Canvas perform well on large visualisation
datasets. This paper was published in 2008 and browser support is anticipated to
increase over time. Therefore as a part of this project it was deemed necessary
to conduct a new set of tests tailored to benchmark network visualisation in the
browser.


  % Pulished implementations Becker 1995, Paul 2000.
Paul, et al. produced a Java 3D implementation of a network weather map and a
performance data poller.\cite{Paul_2000} The map incorporates an idea of
subnetworks having specific layout types such as star, ring and sphere. More
node detail is made available as the user moves through the 3D space. This
solution suffers from portability issues as it is reliant the software being
installed on various machines with connectivity.

Becker, et al. present a geographical network mapping tool.\cite{Becker_1995}
The visualisation is static with controls loctted around the map to look at
different aspects of the data. They struggle with ideas to manage navigation
when zoomed in. They suggest fish eye navigation may be useful to manage such
case. They also look at ways of showing a large network in a compact display
through aggregation of links and geographic omission. However, aggregation
decreases the information about particular links. This tool is quite dated and
is not web based.

  % Fat and thin client work discussion.
  % Where does my project fit in here?


\section{Background}
\label{sec:background}
  % Background section
  % What can't I assume that people will have as a general background
  % Overview of background section???
  %
  % Datasets, Technologies, Libraries

\subsection{Datasets}
\label{sec:datasets}
  %
  % * What datasets did I use?
  % * Why did I choose/use those?
  %   > real network data
  %   > good for evaluation
  % * How did I use those datasets?
  %   > raw networking -> adapter -> generic JSON
  %   > possibly refer implementation section for further detail

Example datasets were required in order to be able to evaluate network map
designs and to support test driven development in the implementation. For this
project, two main datasets were made use of for this purpose. Both of the
datasets describe real networks currently in active service and both networks
have an existing network map tool. This is particularly useful for this project
because we can be sure that we are getting a good representation of an actual
network and it makes for effective evaluation when it is possible to get
comparative feedback from network engineers. 

  % Karen About, PHP weathermap 
The first dataset used was from the Karen network. \cite{Karen_website} Karen is
a high capacity network that links together education and research institutions
throughout New Zealand. Their network consists of points-of-presence (PoPs)
strategically placed in regions in the North and South Island and with
connections to Sydney and Los Angeles. Each PoP can be thought of as a
subnetwork of Karen that may contain distribution devices, and connections to
institutions and other PoPs.  Performance data such as bytes and packets per
second for a given device port is split over a series of round-robin database
(RRD) files. Device information such as name and location are stored in a
database and also available on their online weathermap.

Karen currently uses the network map generation tool called PHP Network
Weathermap. \cite{PHP_Network_Weathermap_website} See Appendix
\ref{app:karenphp} for a snapshot of the map. This tool reads in common network
data files on the server side and generates an image representation of the
network topology. The image that is sent to the client side includes some
interaction through the use of HTML image maps but requires a complete
regeneration for a change of view. This limits interactivity and makes it
difficult for a user to effectively navigate into deeper subnetworks. This
project attempts to address these shortcomings by using techniques described in
Section \ref{sec:visual-design}. 

  % Crcnet
The second dataset was sourced from Rural Link's wireless network based in both
rural and city areas across Waikato. The network's core is located in Hamilton
city and branches out across wireless access points in a tree like structure.
The dataset obtained includes device information and relationships without
performance data. Rural Link uses a network management system called Nagios that
also includes its own network map tool. \cite{Nagio_website} See Appendix
\ref{app:crcnetnagios} for an example of what their current map looks like. This
network mapping tool only supports networks laid out in a tree structure, does
not support interaction of any kind and does not visualise performance data. It
does have a good way of visualising whether or not a link between devices is up
or down by highlighting areas of the tree green or red. 


  % Adapters Why use them. How did I use them. Relate to Datasets TODO: maybe
  % define adapter if used elsewhere??
It is useful keep code dealing with the raw network configuration and
performance data separate from the visualisation. This project used a generic
graph data structure on the visualisation side try to handle any type of map.
The data structure simply defines a set of nodes each with their own set of
adjacencies. Nodes and adjacencies both may have data included which allows for
the addition of any number of parameters. Raw networking data, perhaps stored in
RRDs or other databases, can be processed on the server side by an adapter and
converted into the generic form that the visualisation understands.  The generic
visualisation structure is done in Javascript Object Notation (JSON) which makes
it trivial to transmit between the server and client side. \cite{rfc4627}  For
example, to get the Rural Link configuration files to the generic visualisation
form, a python script was written to act as an adapter between the two formats. 


\subsection{Technologies}
\label{sec:technologies}
  % Go over the key technologies used
  %   > eg Javascript, Canvas, HTML/CSS
  % Why did I choose to use these technologies
  % How did I pick them?
  %   > browser scalability tests
  %     - overview of my testing tool
  %     - show and explain implications of results
  %     - refer back to the scalability paper


\subsection{Libraries}
\label{sec:libraries}
  % Go over the key libraries used
  %   > eg jQuery, Javascript Infovis Toolkit, arbor
  % Why did I choose these libraries
  % How did I pick them?

\section{Visual Design}
\label{sec:visual-design}
  % Overview of visual design

\subsection{Nodes}
\label{sec:nodes}
  % Design considerations for nodes
  % Nodes can be sub networks
  %   > Helps for scalability
  % Grouping, hiding nodes
  % Combining with symantic zooming
  % Screenshots

\subsection{Edges}
\label{sec:edges}
  % Simple line between two nodes
  % Can make use of this space more effectively
  % Explain the visual design of my way
  %   > visualisation principles included
  %   > Quantitative vs qualitative
  % Screenshots

\subsection{Layouts}
\label{sec:layouts}
  % Explain importance of node layout
  %   > Different subgroups - different layouts
  %   > Groups have layouts
  % Which layouts did I implement and why
  % Screenshots

\subsection{Navigation}
\label{sec:navigation}
  % Details of navigation
  % Why is getting it right important
  % Issues relating to navigation
  % Zoom, pan, zoom/center nodes, follow edges
  % Screenshots

\subsection{Overviews}
\label{sec:overviews}
  %
  %
  %

\section{Implementation} 
\label{sec:implementation}

\subsection{Nodes and Edges}
\label{sec:nodes-and-edges}

\subsection{Groups}
\label{sec:groups}

\subsection{Layouts}
\label{sec:layouts}

\subsubsection{Editor}
\label{sec:editor}

\subsection{Navigation}
\label{sec:navigation}

\subsection{Overviews}
\label{sec:overviews}

\subsection{Overlays}
\label{sec:overlays}

\section{Deployment Examples}
\label{sec:deployment-examples}

\section{Evaluation}
\label{sec:evaluation}

\section{Future Work}
\label{sec:future-work}
  % Discuss possible branches of future work
  %   > Incorporate the map as part of a network management system
  %   > Experiment with more layout algorithms
  %   > Look into the posibility of effectively incorporating layers
  %   > Generate adapters to take common network data types such as RRD

\section{Conclusion}
\label{sec:conclusion}

\bibliographystyle{plain}
\bibliography{}

\appendix
\appendixpage

\section{Karen PHP Network Weathermap}
\label{app:karenphp}

\section{Rural Link Nagios Network Map}
\label{app:crcnetnagios}

\end{document}
